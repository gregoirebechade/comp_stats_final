\section{Presentation of the Paper}

\begin{frame}{Objectifs}
Analyse de signaux electroencéphalogrammes (\textbf{EEG}) via l'apprentissage auto-supervisé (\textbf{SSL}).
\begin{itemize}
    \item Réduire les annotations coûteuses.
    \item Modèles entrainés pour identifier la structure des données
    \item Deux problèmes d'applications:
    \begin{itemize}
        \item \textbf{Sleep staging}: Classification des phases de sommeil.
        \item \textbf{Pathology detection}: Détection de pathologies.
    \end{itemize}
    \end{itemize}
\end{frame}

\begin{frame}{Approche}
Tâche prétextuelle :
    \begin{enumerate}
        \item \textbf{Positionnement relatif} (RP) : Prédire si deux fenêtres EEG sont proches dans le temps.
        \item \textbf{Mélange temporel}: (TS) : Vérifier si des fenêtres sont dans l’ordre ou mélangées.
        \item \textbf{Codage prédictif contrastif}: (CPC) : Prédire les fenêtres futures à partir du contexte.
    \end{enumerate}
Représentations avec des réseaux convolutionnels.
\end{frame}

\begin{frame}{Résultats clés}
\begin{itemize}
    \item SSL surpasse les modèles supervisés dans des contextes avec peu de données labelisées.
    \item Les représentations apprises capturent des structures physiologiques pertinentes.
\end{itemize}
\end{frame}

\begin{frame}{Critique des hypothèses}
\begin{itemize}
    \item Structure temporelle des EEG: limites si changements abrupts.
    \item Volume de données non annotées : normalisation robuste pour éviter le bruit.
    \item Généralisation des représentations : âge pour les phases de sommeil.
    \item Tâche prétextuelle : analyses fréquentielles pour compléter les résultats.
\end{itemize}
\end{frame}


