\section{Presentation of the Paper}

\begin{frame}{Objectives}
Analysis of electroencephalogram (\textbf{EEG}) signals using self-supervised learning (\textbf{SSL}).
\begin{itemize}
    \item Reduce costly annotations.
    \item Train models to identify the structure of the data.
    \item Two application problems:
    \begin{itemize}
        \item \textbf{Sleep staging}: Classification of sleep stages.
        \item \textbf{Pathology detection}: Detection of pathologies.
    \end{itemize}
\end{itemize}
\end{frame}

\begin{frame}{Approach}
Pretext task:
    \begin{enumerate}
        \item \textbf{Relative positioning} (RP): Predict whether two EEG windows are close in time.
        \item \textbf{Temporal shuffling} (TS): Verify if windows are in order or shuffled.
        \item \textbf{Contrastive predictive coding} (CPC): Predict future windows from the context.
    \end{enumerate}
Representations learned using convolutional networks.
\end{frame}

\begin{frame}{Key Results}
\begin{itemize}
    \item SSL outperforms supervised models in contexts with limited labeled data.
    \item The learned representations capture relevant physiological structures (apnea, age).
\end{itemize}
\end{frame}

\begin{frame}{Critics of the Hypotheses}
\begin{itemize}
    \item Temporal structure of EEG: limitations if abrupt changes occur.
    \item Pretext task: frequency analyses to complement results.
\end{itemize}
\end{frame}
